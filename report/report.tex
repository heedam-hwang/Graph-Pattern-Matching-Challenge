\documentclass{article}

% \everymath{\displaystyle}
\usepackage{kotex, amsmath, amssymb, amsthm}
\usepackage{biblatex}
\author{2019-15861 염준영, 황희담}

\addbibresource{bib.bib}
\AtEveryCitekey{%
  \ifkeyword{kobib}{%
    \renewcommand{\multinamedelim}{, }%
    \renewcommand{\finalnamedelim}{, }%
  }{%
  }%
}
\AtEveryBibitem{%
  \ifkeyword{kobib}{%
    \renewcommand{\multinamedelim}{·}%
    \renewcommand{\finalnamedelim}{·}%
    \DeclareFieldFormat{journaltitle}{<<#1>>}%
  }{%
  }%
}
\linespread{1.6}
\title{Graph Pattern Matching Challenge}


\begin{document}
\maketitle
\section{알고리즘}
\subsection{Matching Order}
% \cite{한명지2018AEAf}
% \cite{10.1145/3299869.3319880}
% \cite{김현준2020FGQP}

\cite{10.1145/3299869.3319880}에 있는 Weight array 방법(Path size order)을 사용하여 Vertex의 순서를 정하였다.
다만 매번 \(\sum_{v \in C_{M}(u)} W_{u}(v)\)를 계산하는 것보다, 그냥 \(\sum_{v \in C(u)} W_{u}(v)\) 를 계산해 놓고
그 weight대로 매칭하는 것이 실제 성능이 더 잘 나와서 \(\sum_{v \in C(u)} W_{u}(v)\)를 사용하였다.
\verb|DAG::InitWeight| 함수에서 확인할 수 있다.\\
또한, \(u\)를 정한 후, \(v \in C_{M}(u)\)에 대하여 \(M' \leftarrow M \cup \{(u, v)\}\)를 추가한 후
매칭할 때, \(v\)를 선택하는 순서를 \(u\)의 이웃이 가진 라벨 중 가장 빈도가 높은 라벨을 \(l\)이라 할 때,
\(v\)의 이웃이 가진 \(l\)의 수를 기준으로 하여 내림차순으로 매칭하게 하여, 
더 가능성이 높은 구조를 찾아 더 빠르게 매칭되도록 하였다.
(\verb|Backtrack::AdaptiveMatching|함수에서 extendable을 sort하는 부분)

\subsection{Backtracking}
기본적으로 \cite{10.1145/3299869.3319880}, ``Algorithm 2''의 방법을 따른다, 
다만 앞서 Matching order절에서 언급한 바와 같이, \(v \in C_M(u)\)를 순회하며 매칭할 때 
\(C_M(u)\)의 순서를 정렬한 후 탐색한다.


\section{실행 환경}
Linux 및 macOS환경에서, readme에 언급된 바와 같이 cmake와 make를 이용하여 컴파일하였다.
    
\printbibliography
\end{document}